% XeLaTeX can use any Mac OS X font. See the setromanfont command below.
% Input to XeLaTeX is full Unicode, so Unicode characters can be typed directly into the source.

% The next lines tell TeXShop to typeset with xelatex, and to open and save the source with Unicode encoding.

%!TEX TS-program = xelatex
%!TEX encoding = UTF-8 Unicode

\documentclass[12pt]{article}
\usepackage{geometry}                % See geometry.pdf to learn the layout options. There are lots.
\usepackage{listings}
\usepackage{ulem}
\usepackage{xeCJK}
\usepackage{indentfirst}
\usepackage{amsthm}
\usepackage{underscore}  
\usepackage{amsmath}
\geometry{letterpaper}                   % ... or a4paper or a5paper or ... 
%\geometry{landscape}                % Activate for for rotated page geometry
%\usepackage[parfill]{parskip}    % Activate to begin paragraphs with an empty line rather than an indent
\usepackage{graphicx}
\usepackage{amssymb}
\usepackage{geometry}
\geometry{left=2.5cm,right=2.5cm,top=2.5cm,bottom=2.5cm}
\newfontfamily\menlo{Menlo}

% Will Robertson's fontspec.sty can be used to simplify font choices.
% To experiment, open /Applications/Font Book to examine the fonts provided on Mac OS X,
% and change "Hoefler Text" to any of these choices.


\title{Conservative Parallel Detection Algorithm for OpenACC}  
\author{吴昊泽·张煜皓·陈牧歌}
%\date{}                                           % Activate to display a given date or no date

\begin{document}
	\maketitle
	\section{选题概述} 
		OpenACC指令在加速科学代码方面是一种简单而又可移植的方式。 利用 OpenACC,只要在自己的 Fortran 或 C 语言代码中插入编译器提示,编译器即可将代码中计算量繁重的部分自动交由 GPU 处理,以实现更高的性能。\\
		\indent OpenACC指令最大的长处就是简单,只需要插入一句编译器的提示,就可以将代码中的某个语句块进行并行优化。正因如此,我们小组选择的课题就是:针对C代码中的大量循环,通过对程序的静态分析,以检测哪些循环能够并行。\\
	\section{问题抽象}
		\subsection{工具功能}
			由于OpenACC的指令较多,我们小组无法对其进行系统的分析,故我们选取了最有代表,也最关键的\[\text{\#pragma acc parallel loop}\]这条指令进行分析。其指令的作用就是将下一条语句进行并行处理。\\
			\indent 我们工具达到的效果就是分析给定的一段代码后,自动在代码中一定可以并行的地方加入指令;如果无法判断一处是否可以并行,也会输出相应的注释,去帮助OpenACC的使用者更佳方便地优化自己的程序。\\
		\subsection{工具情况}
			我们的工具能够检测大部分情况下的并行情况,但也有部分情况受制于小组水平与静态分析本身的缺陷,无法检测是否并行。在接下来的分析中,可以看到我们的工具是有安全性的保证的。而准确性的不足,也会在后面的部分详细给出。
		\subsection{简化C语言}
			我们的代码分析都是在简化之后的C语言上进行的。之所以需要简化C语言,是因为由于程序静态分析的特殊需要,我们小组写了个针对C语言的解释器,还因为这可以简化我们的工具的制作。但值得注意的是,针对其中的每一条简化,我们都有相应的原因作为支撑,以表明这是一个合理的简化。
			\begin{table}[htbp]
				\caption{简化C语言}
				\center
				\begin{tabular}{|l|l|}
					\hline
					\makebox[5cm][c]{\textbf{简化}} & \makebox[5cm][c]{\textbf{原因}} \\
					\hline\hline
					不支持任何形式的\text{\&}运算&为了提高对于数组分析的精度,不\\
										      &得不对指针做出一点限制\\ \hline
					所有的变量定义要放在一个代码块&早期的C语言标准就是如此,并且\\
					的开头					   &这可以简化解释器的编写\\ \hline
					忽略\text{\#define}语句&这可以简化解释器的编写 \\ \hline
					不允许调用函数			     &暂时做的是过程内的分析,但 可特\\ 
										     &殊处理min,max等简单函数      \\ \hline
					不允许一个语句中修改多个变量&算法上支持,目的仅为简化解释器 \\ 
										     &的编写      \\ \hline
					%TODO
				\end{tabular}			
			\end{table}
	\section{算法安全性证明}
		\subsection{算法简述}
			%TODO可能要写具体parser的流程
			首先,将整个程序parser成一个CFG图,之后在这个CFG上做程序流分析:逆向做一遍程序流分析,分析在该语句之后,有哪些变量是活跃变量;正向做一遍程序流分析,得到每个for循环与表达式的摘要,具体摘要的定义将在接下来给出。对于每个循环,我们通过对这两次分析的结果分析其能否被并行。\\
			\begin{itemize}
				\item 活跃变量分析:
				\begin{itemize}
					\item[1)] 逆向分析
					\item[2)] 半格元素:一个集合,这个操作之前,哪些变量是活跃的。
					\item[3)] 交汇操作:集合的并集
					\begin{item}[4)] 
						变换函数\\
						\begin{description}
							\item[\text{var被修改}]
								$KILL=\{var\},GEN=\varnothing$
							\item[\text{var被使用}]
								$KILL=\varnothing,GEN=\{var\}$
						\end{description}
					\end{item}
				\end{itemize}
				\item 语句摘要分析:
				\begin{itemize} 
					\item[0)] 摘要:5个集合\\
						\begin{description}
							\item[\text{Array\_Modify}]
								表示在这个语句中哪些数组被修改,我们除了记录数组之外,还需要记录其被修改的位置$(i_1,i_2\dots ,i_d)$,其中$d$表示数组的维数,$i_x$为一个表达式。
							\item[\text{Array\_Use}]
								表示在这个语句中哪些数组被使用,同\text{Array\_Modify}的记录一样,需要记录数组与其被使用的位置。
							\item[\text{Var\_Modify}]
								表示在这个语句中有哪些普通变量被修改,如果语句是if,while,for,那么这里的变量指的是在语句之外的变量,而不考虑内部的局部变量。
							\item[\text{Var\_Use}]
								表示在这个语句中有哪些普通变量被使用,同\text{Var\_Modify}的记录一样,对if,while,for语句特殊处理。
							\item[\text{Var\_Reduction}]
								表示在这个语句中有哪些普通变量被修改了,并且记录对它们的操作满足结合律,即可以reduction,集合元素就是\text{\(var,op\)}。
						\end{description}
					\item[1)] 正向分析
					\item[2)] 半格元素:摘要的5个集合
					\item[3)] 交汇操作:摘要集合的并集
					\begin{item}[4)] 
						变换函数\\
						\begin{description}
							\item[\text{if,while,for}] 递归处理,将内部的语句的摘要进行合并,并剔除那些在语句块中的局部变量,作为对这些语句的摘要。
							\item[\text{其他普通语句}]	 将相应的变量(数组变量和普通变量)放入对应的集合之中。并且分析每个被改变的变量的操作是否满足结合律,如果是,也放入\text{Var\_Reduction}集合之中。
						\end{description}
					\end{item}
				\end{itemize}
				\item for循环并行分析:\\
					对于for内部而言,只关注其变量能否并行,变量分为数组变量和普通变量,for内部的局部变量不纳入考虑。
					\begin{description}
						\item[\text{数组}] 对于每个数组a而言,我们去该for语句的摘要中查找其可能被访问或者修改的位置。\\
							\begin{itemize}
								\item[a)] 如果a只被修改,则改数组不影响并行性。
								\item[b)] 如果a只被访问,则改数组不影响并行性。
								\item[c)] 如果a既被修改,也被访问,那么需要知道修改的位置和访问的位置是否有重合的部分。如果有,则不能并行,反之则可以。在这里可以调用SMT Solver求解(详情见3.3准确性的不足)。但是在工具中我们只考虑最简单的可以并行的情况:即改数组的所有下标均是单个循环变量构成的表达式,并且访问和修改处的每个下标构成的表达式完全相等的情况(具体而言:如a[i][j+1]=a[i][j+1]+1,访问的位置为$\text{(i,j+1)}$,修改的位置也为$\text{(i,j+1)}$,其中该访问和修改处的每个下标构成的表达式完全相等)。(安全性证明见3.4安全性证明)
							\end{itemize}
						\item[\text{普通变量}]	 对于每个变量var而言,仅需考虑其在for之外的影响。\\
							\begin{itemize}
								\item[a)] 如果var只被修改,要看在下文中,这个变量是否是活跃的。因为并行的副作用就是导致这个变量在for之行结束后的值不确定,如果活跃,则认为这个for不能并行。
								\item[b)] 如果var只被访问,则改变量不影响并行性。
								\item[c)] 如果var既被修改,也被访问,需要从摘要中得到这个变量是否是reduction的,如果是,则这个变量不影响并行性;如果不是,则不能并行
							\end{itemize}
					\end{description}
			\end{itemize}
		\subsection{安全性证明}
		\subsection{准确性的不足}
	\section{安全性检查}
	\section{小组分工}
		吴昊泽同学负责OpenACC平台的搭建,编写项目工具代码,算法讨论。\\
		\indent 张煜皓同学负责llvm的借口尝试(失败),编写项目报告及少量代码,算法讨论。 \\
		\indent 陈牧歌同学负责对最后工具的测试,算法讨论。
		

% For many users, the previous commands will be enough.
% If you want to directly input Unicode, add an Input Menu or Keyboard to the menu bar 
% using the International Panel in System Preferences.
% Unicode must be typeset using a font containing the appropriate characters.
% Remove the comment signs below for examples.

% \newfontfamily{\A}{Geeza Pro}
% \newfontfamily{\H}[Scale=0.9]{Lucida Grande}
% \newfontfamily{\J}[Scale=0.85]{Osaka}

% Here are some multilingual Unicode fonts: this is Arabic text: {\A السلام عليكم}, this is Hebrew: {\H שלום}, 
% and here's some Japanese: {\J 今日は}.



\end{document}  